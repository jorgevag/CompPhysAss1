\documentclass[11pt,a4paper]{article}
\usepackage[utf8]{inputenc} % Norwegian letters
\usepackage{amsmath}
\usepackage{float}  % Used for minipage and stuff.
\usepackage{wrapfig} % Wrap text around figure wrapfig [tab]
\usepackage{graphicx}
\usepackage{rotating}
\usepackage{enumerate} % Use e.g. \begin{enumerate}[a)]
\usepackage[font={small, it}]{caption} % captions on figures and tables
\usepackage[toc,page]{appendix} % Page make Appendices title, toc fix table of content 
\usepackage{todonotes} % Notes. Use \todo{"text"}. Comment out \listoftodos
\usepackage{microtype} % Improves spacing. Include AFTER fonts
\usepackage{hyperref} % Use \autoref{} and \nameref{}
\hypersetup{backref,
  colorlinks=true,
  breaklinks=true,
  %hidelinks, %uncomment to make links black
  linkcolor=blue,
  urlcolor=blue
}
\usepackage[all]{hypcap} % Makes hyperref jup to top of pictures and tables
%
%-------------------------------------------------------------------------------
% Page layout
%\usepackage{showframe} % Uncomment if you want the margin frames
%\usepackage{fullpage}
\topmargin=-0.25in
%\evensidemargin=-0.3in
%\oddsidemargin=-0.3in
%\textwidth=6.9in
%\textheight=9.5in
\headsep=0.25in
\footskip=0.50in

%-------------------------------------------------------------------------------
% Header and footer
\usepackage{lastpage} % To be able to add last page in footer.
\usepackage{fancyhdr} % Custom headers and footers
%\pagestyle{fancy} % Use "fancyplain" for header in all pages
%\renewcommand{\chaptermark}[1]{ \markboth{#1}{} } % Usefull for book?
\renewcommand{\sectionmark}[1]{ \markright{\thesection\ #1}{} } % Remove formating and nr.
%\fancyhead[LE, RO]{\footnotesize\leftmark}
%\fancyhead[RO, LE]{\footnotesize\rightmark}
%\lhead[]{\AuthorName}
%\rhead[]{\rightmark}
%\fancyfoot[L]{} % Empty left footer
%\fancyfoot[C]{} % Empty center footer
%\fancyfoot[R]{Page\ \thepage\ of\ \protect\pageref*{LastPage}} % Page numbering for right footer
%\renewcommand{\headrulewidth}{1pt} % header underlines
%\renewcommand{\footrulewidth}{1pt} % footer underlines
%\setlength{\headheight}{13.6pt} % Customize the height of the header

%-------------------------------------------------------------------------------
% Suppose to make it easier for LaTeX to place figures and tables where I want.
%\setlength{\abovecaptionskip}{0pt plus 1pt minus 2pt} % Makes caption come closer to figure.
%\setcounter{totalnumber}{5}
%\renewcommand{\textfraction}{0.05}
%\renewcommand{\topfraction}{0.95}
%\renewcommand{\bottomfraction}{0.95}
%\renewcommand{\floatpagefraction}{0.35}
%
% Math short cuts for expectation, variance and covariance
\newcommand{\E}{\mathrm{E}}
\newcommand{\Var}{\mathrm{Var}}
\newcommand{\Cov}{\mathrm{Cov}}
% Commands for argmin and argmax
\DeclareMathOperator*{\argmin}{arg\,min}
\DeclareMathOperator*{\argmax}{arg\,max}
%%%%%%%%%%%%%%%%%%%%%%%%%%%%%%%%%%%%%%%%%%%%%%%%%%%%%%%%%%%%%%%%%%%%%%%%%%%%%%%%
%-----------------------------------------------------------------------------
%	TITLE SECTION
%-----------------------------------------------------------------------------
\newcommand{\AuthorName}{Jørgen Vågan} % Your name
%
\newcommand{\horrule}[1]{\rule{\linewidth}{#1}} % Create horizontal rule command with 1 argument of height
%
\title{	
\normalfont \normalsize 
\textsc{TFY4235 NTNU} \\ [25pt] % Your university, school and/or department name(s)
\horrule{0.5pt} \\[0.4cm] % Thin top horizontal rule
\huge Computational Physics \\ Assignment I \\ % The assignment title
\horrule{2pt} \\[0.5cm] % Thick bottom horizontal rule
}
%
\author{\AuthorName } % Your name
%
\date{\normalsize\today} % Today's date or a custom date
\begin{document}
\maketitle
%%%%%%%%%%%%%%%%%%%%%%%%%%%%%%%%%%%%%%%%%%%%%%%%%%%%%%%%%%%%%%%%%%%%%%%%%%%%%%%%
\listoftodos{}
%
\begin{abstract}

\end{abstract}
%
\section{Introduction}
If one wants to separate particles in a solution, there are several techniques one can use to obtain this. 
Here, considering a constituents of different mass and size, the particles will be separated by the use 
of a perodic asymmetric potential serving as a set of potential wells.
By switching the potential on and off with a certain time period, one can bias the particles\' brownian motion
such that lighter and smaller particles are more likely to cross the potential barriers and jump to the 
neighboring wells than the larger and heavier ones. This is called to use a rachet potential

\section{Modeling of The Problem}
\subsection{Theory}
A small particle placed in a solution will be subject to random collisions with the solvent and
constituent molecules. The force on the particle due to the collisions varies in both force and
direction and causes the particle to move in an unregular and unpredictable pattern
or \textit{Brownian Motion}. 
In addition to the fluctuating force the particles are also subject to viscous drag, such that the extent of the 
particles erratic movement is decided by their mass and size/surface. \\
Now, consider two constituents of spherical particles with radius $r_1$, $r_2$ and masses
$m_1$, $m_2$, respectively. The 1D equation of motion for particle $i$ is given by Newton's $2^{\text{nd}}$ law
\begin{align}
   m_i \frac{d^2x}{dt^2} = \: - \! \underbrace{\frac{\partial}{\partial x} U(x_i,t)}_{\text{potential force}} \:
                           - \!\!\!\! \underbrace{\gamma _i \frac{dx_i}{dt}}_{\text{friction force}} 
                            \! \! \! + \:\:\:  \xi (t),
\end{align}
where $\gamma_i = 6\pi \eta r_i$ is the friction constant, $U$ is the potential energy and $\xi$ is a stochastic variable modeling the particle's brownian motion. Furthermore, the potential is assumed to have the form of
\begin{align}
   U(x,t) = U_r(x) f(t),
\end{align}
%
\begin{align}
    U_r(x)= 
\begin{cases}
    \frac{x}{\alpha L} \Delta U, & \text{if } \: 0 \leq x < \alpha L\\
   \frac{L-x}{L(1-\alpha)} \Delta U, & \text{if } \alpha L \leq x < L
\end{cases}
\!
,
\:\:\:
f(t) =
\begin{cases}
    0, & \text{if } \:\:\: 0 \leq t < \frac{3\tau}{4}\\
   1, & \text{if } \frac{3\tau}{4} \leq t < \tau
\end{cases}
\end{align}
whith $U_r(x)$ being the spatial part, having an asymmetric saw-tooth shape with periodicity L.
The time dependency is described $f(t)$ which is a square wave signal, turning the potential
on and off, with a period of $\tau$. $\alpha\in [1,0]$ is the asymmetry factor, giving how much 
the potential is skewed (with $\alpha = 0.5$ being the symmetric case). \\
%
The resulting iterative numerical scheme, based on the equations above, is a forward Euler given by
\begin{align}
   x_{n+1} = x_n - \frac{1}{\gamma_i} \frac{\partial}{\partial x} U(x_n,t_n) \delta t
                  + \sqrt{ \frac{2 k_B T \delta t}{\gamma_i} } \: \hat{\xi} (t).
\label{euler_scheme}
\end{align}
Choosing 
\begin{align}
   \hat{x} = \frac{x}{L}, \:\:\:\:\:\:\:
   \hat{t} = \omega t,    \:\:\:\:\:\:\:
   \hat{U}(\hat{x},\hat{t})=\frac{U(x,t)}{\Delta U}, \:\:\:\:\:\:\:
   \omega = \frac{\Delta U}{\gamma_i L^2}, \:\:\:\:\:\:\:
   \hat{D} = \frac{k_B T}{ \Delta U}
\end{align}
such that
\begin{align*}
   \frac{\partial}{\partial x} U(x,t)
   &= \frac{\partial}{\partial x} \big( \Delta U \cdot \hat{U}(\hat{x}, \hat{t}) \big) \\
   &= \Delta U \cdot \frac{\partial}{\partial x} \big( \hat{U}_r(\hat{x})\big) \hat{f}(\hat{t}) \\
   &= \Delta U \cdot \frac{\partial \hat{U}_r(\hat{x})}{\partial \hat{x}}  \frac{\partial \hat{x}}{\partial x} \hat{f}(\hat{t}) \\
   &= \frac{\Delta U}{L} \cdot \frac{\partial}{\partial \hat{x}} \hat{U}_r(\hat{x}, \hat{t})
\end{align*}
$\big( \text{note that } f(t) = \hat{f}(\hat{t}) \: \big)$, equation \eqref{euler_scheme} can be rewritten in reduced units
\begin{align}
   \hat{x}_{n+1} = \hat{x}_n - \frac{\partial}{\partial \hat{x}} \hat{U}(\hat{x}_n,\hat{t}_n) \delta \hat{t}
   + \sqrt{ 2 \hat{D} \delta \hat{t} } \: \hat{\xi}.
\label{reduced_euler_scheme}
\end{align}


\textbf{Box-Müller Algorithm} \\
A gaussian probability distribution is used to model the particles brownian motion. The stochastic variable
\todo{can i say the this?} is obtained using the Box-Müller algorithm. Assuming mean $\mu = 0$ and unit
standard deviation the 2D gaussian distribution can be expressed as
\begin{align*}
   p(x) = \frac{1}{2 \pi }e^{-\frac{x^2 + y^2}{2}} = \frac{1}{2 \pi }e^{-\frac{r^2}{2}}
\end{align*}
%REFERENCE:
%http://www.math.nyu.edu/faculty/goodman/teaching/MonteCarlo2005/notes/GaussianSampling.pdf
using cartesian coordinates. Since the distribution is radially symmetric, it is a good choice
to use polar coordinates ($r$,$\phi$), with $x = r \cos \theta$, $y = r \sin \theta$ 
and $\theta \in [0,2\pi]$. Due to the radial symmetry $\theta$ is uniformly distributed on its interval
and may be sampled using
\begin{align*}
   \theta = 2 \pi \xi_2.
\end{align*}
The resulting distribution function as a function of $r$
\begin{align*}
   P(r' \leq r) = P(r) &= \int_0^{2\pi} \int_0^r \frac{1}{2 \pi }e^{-\frac{r'^2}{2}} r' dr' d\theta \\
                       &= \int_0^r e^{-\frac{r'^2}{2}} dr' \\
                       &= 1-e^{-\frac{r^2}{2}},
\end{align*}
which we can rewrite to 
\begin{align*}
   r = \sqrt{-2ln(1-P(r))} = \sqrt{-2ln(1-\xi')} = \sqrt{-2ln \xi_1}.
\end{align*}
By choosing two independent uniformly distributed numbers, 
$\xi_1$,$\xi_2 \in U(0,1)$, setting $P(r) = \xi_1$ and $\theta = 2 \pi \xi_2 $,
\begin{subequations}
\label{box_muller}
\begin{align}
   x &= r \cos \theta = \sqrt{-2ln \xi_1} \cos (2 \pi \xi_2) \label{box_muller}\\
   y &= r \sin \theta = \sqrt{-2ln \xi_1} \sin (2 \pi \xi_2). \label{box_muller}
   \end{align}
\end{subequations}
are two uncorrelated gaussian distributed numbers.


\textbf{Time Step Size} \\
$99.99\%$ of the time, the random number drawn from the gaussian distribution follows that $ |\hat{\xi} | < 4$.
Assuming $0 < \alpha < 1/2$, the shortest length of a constant force region is given by $\alpha L$.
To get a reasonable answer the iterations must satisfy
\begin{align}
   |x_{n+1} -x_n| \ll \alpha L.
\end{align}
By inserting the iteration scheme, Eq.\eqref{reduced_euler_scheme}, the expression can be rewritten
\begin{align*}
   L| \hat{x}_{n+1} - \hat{x}_n| &\ll \alpha L \\
    \Big| - \frac{\partial}{\partial \hat{x}} \hat{U}(\hat{x}_n,\hat{t}_n) \delta \hat{t}
   		+  \sqrt{ 2 \hat{D} \delta \hat{t} } \: \hat{\xi} \: \Big| &\ll \alpha .
\end{align*}
In the shortest region the particle experiences the largest force. Assuming that $|\hat{\xi}| < 4$ together with that the force and the kick act in the same direction, the criterion becomes
\begin{align}
    \text{max}\Big|\frac{\partial}{\partial \hat{x}} \hat{U}(\hat{x}_n,\hat{t}_n)  \Big| 
    	\delta \hat{t}
   			+ 4 \sqrt{ 2 \hat{D} \delta \hat{t} }  \: &\ll \alpha.
\end{align}



\subsection{Implementation}

\section{Results}
\section{Discussion}
\section{References}

%
\end{document}
